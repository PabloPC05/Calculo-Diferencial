\section{Aplicaciones diferenciables}
\begin{definición}[Dirección]
    Llamaremos \textbf{dirección} a un vector $v\in\mathbb{R}^n$. Normalmente de norma $1$. \\
    Si por ejemplo tenemos $n = 1$ tenemos sólo dos direcciones, $v = 1$ y $v = -1$. En cambio para $n > 1$ tenemos infinitas direcciones. En el caso de $\mathbb{R}^2$ las direccionesde norma 1 pueden escribirse de la forma $v = (\cos\theta, \sin\theta)$, con $\theta\in[0,2\pi)$. \\
\end{definición}

\begin{definición}[Recta]
    Llamaremos \textbf{recta} pasando por $x_0$ y de dirección $v$ a la recta $x(t) = x_0 + tv$, donde $t\in\mathbb{R}$. 
\end{definición}

\begin{definición}[Derivada direccional]
    Si $f$ es una función definida en un subconjunto abierto $A$ de $\mathbb{R}^n$, $x_0$ es un punto de $A$ y $v$ es una dirección de $\mathbb{R}^n$, se define la derivada de $f$ en $x_0$ en la dirección $v$ como
    $$ D_vf(x_0) = \lim_{t\to 0} \frac{f(x_0 + tv) - f(x_0)}{t} $$
\end{definición}
\begin{observación}
    Para cualquier dirección $v$ tanto ella como su opuesta $-v$ definen lamisma recta pasando por $x_0$ (el vector de dirección también determina una orientación). Sin embargo, las derivadas en las direcciones $v$ y $-v$ son de signo opuesto: 
    \begin{align*}
        D_{-v}f(x_{0})
        &= \lim_{t \to 0} \frac{f\bigl(x_{0} + t(-v)\bigr) - f(x_{0})}{t} \\
        &= \lim_{t \to 0} \frac{f\bigl(x_{0} + (-t)\,v\bigr) - f(x_{0})}{t} \\
        &= \lim_{s \to 0} \frac{f\bigl(x_{0} + s\,v\bigr) - f(x_{0})}{-s} \\
        &= - \lim_{s \to 0} \frac{f\bigl(x_{0} + s\,v\bigr) - f(x_{0})}{s} \\
        &= -\,D_{v}f(x_{0}) \,.
    \end{align*}
\end{observación}

\begin{definición}[Derivadas parciales]
    Consideemos en $\mathbb{R}^m$ las direccionesdadas porlos vectores $e_i = (0,\ldots,0,1,0,\ldots,0)$, donde $1$ está en la $i$-ésima posición. 
    Las derivadas en un punto $x_0$ de una función $f$ en estas direcciones (si es que existen) se llaman \textbf{derivadas parciales} de $f$ en $x_0$ y se denotan por $D_if(x_0) = D_{e_i}f(x_0)$, o también por $\frac{\partial f}{\partial x_i}(x_0)$ o $f_{x_i}(x_0)$. 
\end{definición}

\begin{definición}[Gradiente]
    Dada una función $f:A \subset \mathbb{R}^n \to \mathbb{R}$, que tenga todas las derivadas parciales en un punto $x_0\in A$, se llama \textbf{gradiente} de $f$ en $x_0$ al vector
    $$ \nabla f(x_0) = \bigl(D_1f(x_0), D_2f(x_0), \ldots, D_nf(x_0)\bigr) = \bigl(\frac{\partial f}{\partial x_1}(x_0), \frac{\partial f}{\partial x_2}(x_0), \ldots, \frac{\partial f}{\partial x_n}(x_0)\bigr) \in \mathbb{R}^n $$
\end{definición}

\begin{definición}[Diferenciable]
    Dada una función $f:A \subset \mathbb{R}^n \to \mathbb{R}$, es \textbf{diferenciable} en un punto $x_0\in A$ si existe una aplicación lineal $L:\mathbb{R}^n\to\mathbb{R}$ tal que
    $$ \lim_{x\to x_0} \frac{f(x) - f(x_0) - L(x-x_0)}{\|x-x_0\|} = 0 $$
    Lo cual equivale: 
    $$ \lim_{h \to 0} \frac{f(x_0 + h) - f(x_0) - L_{x_0}(h)}{\|h\|} = 0 $$
\end{definición}

\begin{observación}
    La existencia del gradiente no garantiza la diferenciabilidad de la función. 
\end{observación}

\begin{proposición}
    Toda aplicaciónlineal es diferenciable $\forall x \in \mathbb{R}^n$ y su gradiente es la aplicación lineal misma.
\end{proposición}

\begin{teorema}
    Si $f: A \subset \mathbb{R}^n \to \mathbb{R}$ es diferenciable en $x_0\in A$, entonces es derivable en $x_0$ en todas las direcciones $v\in\mathbb{R}^n$. Además sea $v \in \mathbb{R}^n$ una dirección, que supondremos de norma $1$, y una aplicación lineal $L:\mathbb{R}^n\to\mathbb{R}$ tal que
    $$\lim_{h \to 0} \frac{f(x_0 + h) - f(x_0) - L_{x_0}(h)}{\|h\|} = 0$$
    entonces
    $$ L(v) = D_vf(x_0) = \nabla f(x_0) \cdot v $$
\end{teorema}
\begin{proof}
    Tomando sólo vectores $h$ de la forma $tv$ con $t \in \mathbb{R}$, tenemos que
    \[
    \lim_{t \to 0} \frac{f(x_0 + tv) - f(x_0) - tL(v)}{|t|} = 0
    \]
    (\text{nótese que} $\|v\| = 1$) con lo que, con el mismo argumento que hemos utilizado al principio de esta sección, obtenemos que
    \[
    \lim_{t \to 0} \frac{f(x_0 + tv) - f(x_0) - tL(v)}{t} = 0
    \]
    y por lo tanto que
    \[
    \lim_{t \to 0} \frac{f(x_0 + tv) - f(x_0)}{t} = L(v),
    \]
    luego $f$ es derivable en $x_0$ en la dirección $v$ y $D_v f(x_0) = L(v)$.
\end{proof}

\begin{definición}[Diferencial]
    Para represemtar la función $L$ usaremos la notación $df_{x_0}$, que se llama \textbf{diferencial} de $f$ en $x_0$. \\
\end{definición}
\begin{observación}
    Cuando $f$ es diferenciable en $x_0$:
    $$ \frac{\partial f}{\partial x_i}(x_0) = D_{i}f(x_0) = df(x_0)(e_i) = \nabla f(x_0) \cdot e_i $$
    donde $e_i$ es el vector de dirección en la $i$-ésima coordenada. 
\end{observación}

\begin{corolario}
    Sea $f: A \subset \mathbb{R}^n \to \mathbb{R}$ una aplicación diferenciable en un punto $x_0 \in A$ tal que $\nabla f(x_0) \neq 0$ . Entonces el valor máximo de $|D_vf(x_0)|$ se alcanza para $v = \frac{\nabla f(x_0)}{\|\nabla f(x_0)\|}$ y ese valor máximo es $\|\nabla f(x_0)\|_2$.
\end{corolario}
\begin{proof}
    Sabemos que la derivada direccional de $f$ en el punto $x_0$ y en la dirección del vector unitario $v$ viene dada por
    \[
    D_v f(x_0) = \langle \nabla f(x_0), v \rangle.
    \]
    Por la desigualdad de Cauchy-Schwarz, se tiene que
    \[
    |\langle \nabla f(x_0), v \rangle| \leq \|\nabla f(x_0)\|_2 \|v\|_2 = \|\nabla f(x_0)\|_2,
    \]
    ya que $v$ es unitario ($\|v\|_2 = 1$).

    El valor máximo se alcanza cuando $v$ tiene la misma dirección que $\nabla f(x_0)$, es decir,
    \[
    v = \frac{\nabla f(x_0)}{\|\nabla f(x_0)\|_2}.
    \]
    En ese caso,
    \[
    D_v f(x_0) = \left\langle \nabla f(x_0), \frac{\nabla f(x_0)}{\|\nabla f(x_0)\|_2} \right\rangle
    = \frac{\|\nabla f(x_0)\|_2^2}{\|\nabla f(x_0)\|_2} = \|\nabla f(x_0)\|_2.
    \]
    Por lo tanto, el valor máximo de $|D_v f(x_0)|$ es $\|\nabla f(x_0)\|_2$ y se alcanza precisamente en la dirección del gradiente.
\end{proof}

\begin{definición}[Espacio afín tangente]
    Cuando $f$ es diferenciable en $x_0$ llamaremos \textbf{espacio afín tangente} a la gráfica $\{(x,f(x)) : x \in A\} \subset \mathbb{R}^{n+1}$ en el punto $(x_0,f(x_0))$ a
    $$ T = \{(x,f(x_0)) : \langle \nabla f(x_0), x - x_0\rangle : x\in \mathbb{R}^n\} = \{(x, f(x_0)) + df(x_0)(x - x_0)\}$$
\end{definición}
\begin{observación}
    Los espacis afines tangentes son hiperplanos
\end{observación}

\begin{observación}
    El concepto intuitivo de tangencia que tenemos para el caso de curvas en $\mathbb{R}^2$ mantiene también para superficies en $\mathbb{R}^n$ pues si $T$ es el espacio afín tangente a la superficie $\{(x,f(x)) : x \in A\}$ en el punto $(x_0,f(x_0))$, entonces para cada punto $(x, f(x)) : x \in A$ existe un punto $(x, y) \in T$ tal que
    $$ \lim_{x \to x_0} \frac{f(x) - y}{\|x - x_0\|} = 0 $$
    basta tomar $y = f(x_0) + \langle \nabla f(x_0), x - x_0\rangle$.
\end{observación}


\begin{teorema}
    Si $f: A \subset \mathbb{R}^n \to \mathbb{R}$ es diferenciable en un punto $x_0$ entonces $f$ es continua en $x_0$.
\end{teorema}

\begin{proof}
    Escribamos 
    $$f(x) - f(x_0) = f(x) - f(x_0) - \langle \nabla f(x_0), x - x_0\rangle + \langle \nabla f(x_0), x - x_0\rangle$$
    y consideremos el límite
    $$ \lim_{x \to x_0} \frac{f(x) - f(x_0) - \langle \nabla f(x_0), (x - x_0) \rangle}{\|x - x_0\|}  = 0 $$
    resulta que 
    $$ \exists \|x - x_0\| < \delta_1 \implies \left| f(x) - f(x_0) - \langle \nabla f(x_0), (x - x_0) \rangle \right| \leq \|x - x_0\|$$
    Dado que por Cauchy-Schwarz $\|\langle \nabla f(x_0), x - x_0 \rangle \| \leq \|\nabla f(x_0)\| \|x - x_0\|$, podemos escribir
    $$ |f(x) - f(x_0)| < (1 + \|\nabla f(x_0)\|) \|x - x_0\| $$
    si $\|x - x_0\| < \delta_1$. Luego si dado $\epsilon > 0$ tomamos $\delta = \min\{\delta_1, \frac{\epsilon}{1 + \|\nabla f(x_0)\|}\}$, tenemos que si $\|x - x_0\| < \delta$ entonces $|f(x) - f(x_0)| < \epsilon$.
\end{proof}

\begin{teorema}
    Sea $f:A \subset \mathbb{R}^n \to \mathbb{R}$ una aplicación que tiene derivadas parciales en cada punto de $A$. Si para cada $i = 1, \ldots, n$ la función
    $$df: x \in A \mapsto \frac{\partial f}{\partial x_i}(x)$$
    es continua en un punto $x_0 \in A$, entonces $f$ es diferenciable en $x_0$ 
\end{teorema}

\begin{proof}
    Queremos ver si $$ \lim_{x \to x_0} \frac{f(x) - f(x_0) - \langle \nabla f(x_0), x_0 - x \rangle}{\|x - x_0\|} = 0 $$
    Sea una función de la forma $\varphi_i(x) = f(x_{0_1}, \ldots, x_{0_{i-1}}, x_i, x_{0_{i+1}}, \ldots, x_{0_n})$ para $i = 1, \ldots, n$ y fijemos el punto $x_0 \in A$ en el que todas las derivadas de $f$ son continua y consideremos un $r > 0$ tal que $B(x_0, r) \subset A$. Entonces, para cada punto $x \in B(x_0, r)$, tenemos que
    $$f(x) - f(x_0) = f(x_1, \ldots, x_n) - f(x_{0_1}, \ldots, x_{0_n}) = $$
    $$ = f(x_1, x_2, \ldots, x_n) - f(x_{0_1}, x_2, \ldots, x_n) + $$ $$ + f(x_{0_1}, x_2, \ldots, x_n) - f(x_{0_1}, x_{0_2}, \ldots, x_n) + $$ $$+ \ldots + f(x_{0_1}, x_{0_2}, \ldots, x_n) - f(x_{0_1}, x_{0_2}, \ldots, x_{0_n})$$ $$ = f(x) - \varphi_1(x) + \varphi_1(x) - \varphi_2(x) + \varphi_2(x) + \ldots + \varphi_{n}(x) - \varphi_n(x) - f(x_0)$$

    Entonces, apliquemos el Teorema de Valor Medio a $\varphi_1(x)$: $\varphi_1(s) = f(s, x_{0_2}, \ldots, x_{0_n})$ para $s \in [x_{0_1}, x_1]$ es continua y derivable por lo que debe existir un punto $u_1 \in (x_{0_1}, x_1)$ tal que
    $$\varphi_1(x_1) - \varphi_1(x_{0_1}) = \varphi_1'(u_1)(x_1 - x_{0_1})$$
    Si $x_1 = x_{0_1}$ pasamos a la siguiente coordenada, pues en esta primerala diferencia es nula. 
    Pero además tenemos que:
    $$\varphi_1'(u_1) = \frac{\partial f}{\partial x_1}(u_1, x_{0_2}, \ldots, x_{0_n})$$
    Por lo que,
    $$f(x_1, x_2, \ldots,x_n) - f(x_{0_1}, x_2, \ldots, x_n) = D_1f(u_1, x_{0_2}, \ldots, x_{0_n})(x_1 - x_{0_1})$$
    Repitiendo el proceso, conseguimos la existencia de un vector $u = (u_1, u_2, \ldots, u_n)$ tales que 
    $$f(x_{0_1}, \ldots, x_{0_{i-1}}, x_i, x_{i+1}, \ldots, x_n) - f(x_{0_1}, \ldots, x_{0_{i-1}}, x_{0_i}, x_{i+1}, \ldots, x_n) = D_if(x_{0_1}, \ldots, x_{0_{i-1}}, u_i,x_{i+1}, \ldots, x_n)(x_i - x_{0_i})$$
    Tenemos entonces que, 
    $$f(x) - f(x_0) = \sum_{i = 1}^{n} D_if(x_{0_1}, \ldots, x_{0_{i-1}}, u_i,x_{i+1}, \ldots, x_n)(x_i - x_{0_i})$$
    Ahora volvamos al límite que queríamos calcular:
    $$ \lim_{x \to x_0} \frac{f(x) - f(x_0) - \langle \nabla f(x_0), x - x_0 \rangle}{\|x - x_0\|}$$
    Sustiuyamos la expresión que hemos obtenido para $f(x) - f(x_0)$:
    $$ \frac{f(x) - f(x_0) - \langle \nabla f(x_0), x - x_0 \rangle}{\|x - x_0\|} = $$ 
    $$ = \frac{D_1(u_1, \ldots, x_n)(x_1 - x_{0_1}) + \ldots + D_{n}f(x_{0_1}, \ldots, u_n)(x_n - x_{0_n}) - [ D_1f(x_0)(x_1-x_{0_1}) + \ldots + D_nf(x_n)(x_n - x_{0_n})]}{\|x - x_0\|} $$
    $$ \leq \frac{|D_1f(x_{0_1}, \ldots, u_n)(x_1 - x_{0_1}) - D_1f(x_0)(x_1-x_{0_1})| + \ldots + |D_nf(x_{0_1}, \ldots, u_n)(x_n - x_{0_n}) - D_nf(x_0)(x_n - x_{0_n})|}{\|x - x_0\|}$$
    $$ \leq \frac{(D_1f(u_1, \ldots, x_n) - D_1f(x_0))(x - x_0)}{\|x - x_0\|} + \ldots + \frac{(D_nf(x_{0_1}, \ldots, u_n) - D_nf(x_0))(x - x_0)}{\|x - x_0\|} = $$
    $$ = (D_1f(u_1, \ldots, x_n) - D_1f(x_0)) + \ldots + (D_nf(x_{0_1}, \ldots, u_n) - D_nf(x_0)) \xrightarrow{\hspace{0.5cm} x \to x_0 \hspace{0.5cm}} 0$$
\end{proof}

\begin{observación}
    Si $f: A \subset \mathbb{R}^n \to \mathbb{R}^m$ es tal que $\exists df(x) \forall x \in A$, entonces se puede hablar de la función diferencial $df: A \subset \mathbb{R}^n \to \mathbb{R}^{m\times n}$, que es una aplicación lineal en cada punto $x \in A$ 
\end{observación}

\begin{definición}[Matriz Jacobiana]
    Si $f: A \subset \mathbb{R}^n \to \mathbb{R}^m$ es una aplicación diferenciable, entonces la matriz $J_f(x)$ de la función diferencial $df(x)$ se llama \textbf{matriz jacobiana} de $f$ en el punto $x$.
    Si $f$ tiene derivadas parciales en $x$, entonces la matriz jacobiana es la matriz cuyas entradas son las derivadas parciales de $f$ en $x$:
    $$ J_f(x) = \begin{pmatrix}
        \frac{\partial f_1}{\partial x_1}(x) & \frac{\partial f_1}{\partial x_2}(x) & \cdots & \frac{\partial f_1}{\partial x_n}(x) \\
        \frac{\partial f_2}{\partial x_1}(x) & \frac{\partial f_2}{\partial x_2}(x) & \cdots & \frac{\partial f_2}{\partial x_n}(x) \\
        \vdots & \vdots & \ddots & \vdots \\
        \frac{\partial f_m}{\partial x_1}(x) & \frac{\partial f_m}{\partial x_2}(x) & \cdots & \frac{\partial f_m}{\partial x_n}(x)
    \end{pmatrix}$$
    donde $f = (f_1, f_2, \ldots, f_m)$ y $x = (x_1, x_2, \ldots, x_n)$.
\end{definición}

\begin{observación}
    Para aplicaciones lineales $L: \mathbb{R}^n \to \mathbb{R}^m$, se verifica que existe $C > 0$ tal que
    $$ \|L(x)\| \leq C\|x\| \quad \forall x \in \mathbb{R}^n $$
\end{observación}
\begin{proposición}[Propiedades de la matriz jacobiana]
    Sea $f: A \subset \mathbb{R}^n \to \mathbb{R}^m$ una aplicación diferenciable. Entonces:
    \begin{enumerate}
        \item $d(f + g)(x_0) = df(x_0) + dg(x_0)$ para $f, g: A \subset \mathbb{R}^n \to \mathbb{R}^m$ diferenciables en $x_0$.
        \item $d(\alpha f)(x_0) = \alpha df(x_0)$ para $\alpha \in \mathbb{R}$ y $f: A \subset \mathbb{R}^n \to \mathbb{R}^m$ diferenciable en $x_0$.
        \item $d(f \cdot g) = g(x_0) df(x_0) + f(x_0) dg(x_0)$ para $f, g: A \subset \mathbb{R}^n \to \mathbb{R}$ diferenciables en $x_0$.
    \end{enumerate}
\end{proposición}

\begin{teorema}
    Sea $f: A \subset \mathbb{R}^n \to \mathbb{R}^m$ una aplicación diferenciable en un punto de $x_0 \in A$. Entonces existen constantes $M > 0$ y $\delta > 0$ tales que
    $$ \|x - x_0\| < \delta \implies \|f(x) - f(x_0) \| \leq M \|x - x_0\| $$
    En particular, esto implica la continuidad de $f$ en $x_0$.
\end{teorema}
\begin{proof}
    Dado que $f$ es diferenciable se cumple que:
    $$\lim_{x \to x_0} \frac{f(x) - f(x_0) - df(x_0)(x - x_0)}{\|x - x_0\|} = 0$$
    Sea $\epsilon = 1 \implies \exists \delta_1 > 0$ tal que si $\|x - x_0\| < \delta_1$ entonces
    $$0 < \|x - x_0\| < \delta_1 \implies \left|f(x) - f(x_0) - df(x_0)(x - x_0) \right| < \|x - x_0\|$$
    Con lo que 
    $$f(x) - f(x_0) = df(x_0)(x- x_0) + \left[f(x) - f(x_0) - df(x_0)(x -x_0)\right] \implies$$
    $$|f(x) - f(x_0)| = |df(x_0)(x- x_0) + \left[f(x) - f(x_0) - df(x_0)(x -x_0)\right]|$$
    $$\|f(x)-f(x_0)\| \leq \|df(x_0)(x - x_0)\| + \|x - x_0\| \leq |df(x_0)(x- x_0)| + |f(x) - f(x_0) - df(x_0)(x - x_0)|$$
    Pero por la hipótesis tenemos que 
    $$|f(x) - f(x_0) - df(x_0)(x - x_0)| < \|x - x_0\|$$
    Entonces $$\|f(x) - f(x_0)\| \leq |df(x_0)(x- x_0)| + \|x - x_0\|$$
    Pero como $df(x_0)$ es una aplicación lineal, existe una constante $C > 0$ tal que
    $$\|df(x_0)(x - x_0)\| \leq C \|x - x_0\|$$
    Entonces tenemos que $\forall x \in \mathbb{R}^n$ se cumple que
    $$\|f(x) - f(x_0)\| \leq M\|x -x_0\|$$
    para los $x$ tales que $\|x - x_0\| < \delta_1$, donde $M = C + 1$.
\end{proof}

\begin{teorema}[Regla de la cadena]
    Sea $A, \subset \mathbb{R}^n$ y $B \subset \mathbb{R}^m$ abiertos y $f: A \to \mathbb{R}^m$ y $g: B \to \mathbb{R}^p$ tales que $f(A) \subset B$, f es diferenciable en $x_0 \in A$ y $g$ es diferenciable en $f(x_0)$. Entonces $g \circ f$ es diferenciable en $x_0$ y
    $$ d(g \circ f)(x_0) = dg(f(x_0)) \circ df(x_0) $$
    A nivel de matrices, ésto es equivalente a que
    $$ J_{g \circ f}(x_0) = J_g(f(x_0)) \cdot J_f(x_0) $$
\end{teorema}
\begin{proof}
    Tenemos que 
    $$ \frac{\|g(f(x)) - g(f(x_0)) - dg(f(x_0)) \circ df(x_0)(x - x_0)\|}{\|x - x_0\|} \leq $$ 
    
    $$\leq \frac{\|g(f(x)) - g(f(x_0)) - dg(f(x_0))(f(x) - f(x_0))\|}{\|x - x_0\|} + \frac{\|dg(f(x_0))(f(x) - f(x_0) - df(x_0)(x - x_0))\|}{\|x - x_0\|}$$
    Al ser $f$ diferenciable en $x_0$ por el teorema anterior, $M > 0$ y $\delta_1 > 0$ tales que 
    $$ \|x - x_0\| < \delta_1 \implies \|f(x) - f(x_0)\| \leq M\|x - x_0\|$$
    Por otra parte tenemos que al ser $g$ diferenciable en $f(x_0)$, dado $\frac{\epsilon}{2M} > 0$ existe $\delta_2 > 0$ tal que
    $$0 < \|y - f(x_0)\| < \delta_2 \implies \frac{\|g(y) - g(f(x_0)) - dg(f(x_0))(y - f(x_0))\|}{\|y - f(x_0)\|} < \frac{\epsilon}{2M}$$
    Y esto se cumple $\forall y \in B$, en particular para $y = f(x)$. \\
    Tomando $delta_3 = \min\{\delta_1, \delta_2\}$ tenemos que
    $$ 0 < \|x - x_0\| < \delta_3 \implies \frac{\|g(f(x)) - g(f(x_0)) - dg(f(x_0))(f(x) - f(x_0))\|}{\|x - x_0\|} = $$ 
    $$= \frac{\|g(f(x)) - g(f(x_0)) -dg(f(x_0))(f(x) - f(x_0))\| \|f(x) - f(x_0)\|}{\|x - x_0\| \|f(x) - f(x_0)\|} \leq \frac{\epsilon}{2M} \cdot \frac{\|f(x) - f(x_0)\|}{\|x - x_0\|} \leq \frac{\epsilon}{2M} \cdot M = \frac{\epsilon}{2}$$
    Además, por ser $dg(f(x_0))$ una aplicación lineal, existe $C^* > 0$ tal que $\|dg(f(x_0))(y)\| \leq C^* \|y\|$ para todo $y \in \mathbb{R}^m$. \\\\Además, por ser $f$ diferenciable en $x_0$, existe $\delta_4 > 0$ tal que 
    $$ 0 < \|x - x_0\| < \delta_4 \implies \frac{\|f(x) - f(x_0) - df(x_0)(x - x_0)\|}{\|x - x_0\|} < \frac{\epsilon}{2C^*} \implies$$

    $$\frac{\|df(x_0)\left[f(x) - f(x_0)- df(x_0)(x -x_0)\right]\|}{\|x - x_0\|} \leq \frac{C^* \cdot \|f(x) - f(x_0) - df(x_0)(x - x_0)\|}{\|x - x_0\|} \leq \frac{C^*}{2} \cdot \frac{\epsilon}{C^*} = \frac{\epsilon}{2}$$

    Tomando $\delta = \min\{\delta_3, \delta_4\}$ y tomando $x$ tales que $0 < \|x - x_0\| <\delta$ resulta que 
    $$\frac{\|g(f(x)) - g(f(x_0)) - dg(f(x_0)) \circ df(x_0)(x - x_0)\|}{\|x - x_0\|} \leq$$
    $$ \leq \frac{\| g(f(x)) - g(f(x_0)) - dg(f(x_0))(f(x) - f(x_0))\|}{\|x - x_0\|} + \frac{\|dg(f(x_0))(f(x) - f(x_0) - df(x_0)(x - x_0))\|}{\|x - x_0\|} < \frac{\epsilon}{2} + \frac{\epsilon}{2} = \epsilon$$
\end{proof}

\begin{definición}[Convexidad]
    Se dice que un subconjunto $S \subset \mathbb{R}^n$ es convexo si para todo par de puntos $x, y \in S : x \neq y$ se verifica que el segmento $L[x, y] = \{x + t(y- x) : t \in [0,1]\} = \{xt+ y(t- 1) : t \in [0,1]\}$ de extremos x e y está contenido en $S$
\end{definición}

\begin{observación}
    Las bolas son conjuntos convexos con lo que $L[x, y] \subset B(x_0, r)$ 
\end{observación}

\begin{teorema}[Teorema del Valor Medio]
    Sea $f: A \subset \mathbb{R}^n \to \mathbb{R}^m$ una aplicación diferenciable en cada punto de $A$. Entonces para todo $x, y \in A$ tales que $L[x, y] \subset A$ y para todo $z \in \mathbb{R}^m$ se verifica que existe $c \in L[x, y]$ tales que: 
    $$\langle z, f(y) - f(x) \rangle = \langle z, df(c)(y - x) \rangle$$
    En principio, $c$ depende de $x, y, z$ 
\end{teorema}
\begin{proof}
    Como $A$ es abierto $\implies \exists \delta > 0 : (1 - t)x + ty \in A \forall t \in [-\delta, \delta + 1]$. \\
    Fijemos $z \in \mathbb{R}^m$ un vector cualquiera, definamos la función
    $$\varphi(t) = \langle z, f((1 - t)x + ty) \rangle : t \in (-\delta, \delta + 1)$$
    Esta función es derivable en $(-\delta, \delta + 1)$ y 
    $$\varphi'(t) = \langle z, df((1 - t)x + ty)((y - x)) \rangle$$
    pues 
    $$\varphi(t) = z_1 f_1((1 - t)x + ty) + z_2 f_2((1 - t)x + ty) + \ldots + z_m f_m((1 - t)x + ty)$$
    téngase en cuenta que se trata de la diferencial de la composicion de $f$ con la función lineal $g(t) = (1 - t)x + ty$. Luego por el Teorema del Valor Medio para funciones en $\mathbb{R}$ existe $s \in (0,1)$ tal que 
    $$\varphi(1) - \varphi(0) = \varphi'(s)(1 - 0)$$
    de donde se sigue que:
    $$ \langle z, f(y) - f(x) \rangle = \langle z, f(y) \rangle - \langle z, f(x) \rangle = \varphi(1) - \varphi(0) = \varphi'(s) = \langle z, df((1 - s)x + sy)(y - x) \rangle = \langle z, df(c)(y - x) \rangle$$
    Basta tomar $c = (1 - s)x + sy$ para concluir la demostración. 
\end{proof}

\begin{teorema}[Teorema del valor medio para funciones $\mathbb{R}^n \to \mathbb{R}$]
    Sea $f: A \subset \mathbb{R}^n \to \mathbb{R}$ una función diferenciable en cada punto de $A$. Entonces para todo $x, y \in A$ tales que $L[x, y] \subset A$ existe $c \in L[x, y]$ tales que
    $$f(y) - f(x) = df(c)(x - y)$$    
\end{teorema}
\begin{proof}
    Como en este caso $m = 1$ si tomamos $z = 1$ el resultado se sigue directamente del Teorema del Valor Medio anterior, pues
    $$f(y) - f(x) = \langle 1, f(y) - f(x) \rangle = \langle 1, df(c)(y - x) \rangle = df(c)(y - x)$$
\end{proof}

\begin{teorema}[Desigualdad del valor medio para aplicaciones de $\mathbb{R}^n \to \mathbb{R}^m$ o Teorema de los incrementos finitos]
    Sea $f: A \subset \mathbb{R}^n \to \mathbb{R}^m$ una aplicación diferenciable en $A$. Entonces para todo $x, y \in A$ tales que $L[x, y] \subset A$ se verifica que
    $$\|f(y) - f(x)\| \leq \sup_{c \in L[x, y]} \|df(c)\| \cdot \|(y - x)\|$$
\end{teorema}
\begin{proof}
    Sea $x, y \in A : L[x, y] \subset A$ y sea $z = \frac{f(y) - f(x)}{\|f(y) - f(x)\|}$, vector unitario. Por el Teorema del Valor Medio para funciones de $\mathbb{R}^n \to \mathbb{R}$, existe $c \in L[x, y]$ tal que
    $$\langle z, f(y) - f(x) \rangle = \langle z, df(c)(y - x) \rangle$$
    De donde se sigue que 
    $$\|f(y) - f(x)\| = \frac{\|f(y) - f(x)\|^2}{\|f(y) - f(x)\| } = \langle \frac{f(y) - f(x)}{\|f(y) - f(x)\|}, f(y) - f(x) \rangle = \langle \frac{f(y) - f(x)}{\|f(y) - f(x)\|}, df(c)(y - x) \rangle \leq$$
    $$\leq \|\frac{f(y) - f(x)}{\|f(y) - f(x)\|}\| \|df(c)(y - x)\| = \|df(c)(y- x)\| \leq \|df(c)\| \cdot \|y - x\| \leq \sup_{c \in L[x, y]} \|df(c)\| \cdot \|(y - x)\|$$
\end{proof}

\begin{teorema}
    Sea $f: A \subset \mathbb{R}^n \to \mathbb{R}^m$ una aplicación diferenciable en $A$. Entonces para todo $x, y \in A$ tales que $L[x, y] \subset A$ se verifica que
    $$f_j(y) - f_j(x) = df_j(c_j)(y - x)$$
    para algún $c_j \in L[x, y]$ y para cada $j = 1, \ldots, m$.
\end{teorema}
\begin{corolario}
    Sea $A \subset \mathbb{R}^n$ abierto y convexoy $f: A \subset \mathbb{R}^m$ es diferenciable en $A$ y $df(x) = 0 \forall x \in A$. Entonces $f$ es constante en $A$.
\end{corolario}
\begin{definición}[Derivada parciales segundo]
    Sea $f: A \subset \mathbb{R}^n \to \mathbb{R}$ una aplicación diferenciable en $A$. Entonces la \textbf{derivada parcial segunda} de $f$ respecto a la variable $x_i$ es
    $$\frac{\partial^2 f}{\partial x_i^2}(x) = \frac{\partial}{\partial x_i}\left(\frac{\partial f}{\partial x_i}(x)\right)$$
    y la \textbf{derivada parcial mixta} de $f$ respecto a las variables $x_i$ y $x_j$ es
    $$\frac{\partial^2 f}{\partial x_i \partial x_j}(x) = \frac{\partial}{\partial x_j}\left(\frac{\partial f}{\partial x_i}(x)\right)$$
\end{definición}

\begin{definición}[Clase $C^1$]
    Sea $A \subset \mathbb{R}^n$ un conjunto abierto. Una aplicación $f: A \to \mathbb{R}$ se dice que es de clase $C^1$ si tiene derivadas parciales en cada punto de $A$ y estas son continuas en $A$. Si $f: A \to \mathbb{R}^m$ es una aplicación de clase $C^1$, entonces se dice que es de clase $C^1$ si cada componente $f_i$ es de clase $C^1$.
\end{definición}

\begin{definición}[Clase $C^2$]
    Sea $A \subset \mathbb{R}^n$ un conjunto abierto. Una aplicación $f: A \to \mathbb{R}$ se dice que es de clase $C^2$ si tiene derivadas parciales de orden 2 en cada punto de $A$ y estas son continuas en $A$. Si $f: A \to \mathbb{R}^m$ es una aplicación de clase $C^2$, entonces se dice que es de clase $C^2$ si cada componente $f_i$ es de clase $C^2$.
\end{definición}

\begin{definición}[Clase $C^p$]
    Sea $A \subset \mathbb{R}^n$ un conjunto abierto. Una aplicación $f: A \to \mathbb{R}$ se dice que es de clase $C^p$ si tiene derivadas parciales de orden $k$ en cada punto de $A$ y estas son continuas en $A$ para todo $k = 1, \ldots, p$. Si $f: A \to \mathbb{R}^m$ es una aplicación de clase $C^p$, entonces se dice que es de clase $C^p$ si cada componente $f_i$ es de clase $C^p$.
\end{definición}

\begin{lema}
    Sea $f: A \subset \mathbb{R}^n \to \mathbb{R}$ una función de clase $C^2$. Fijado $x_0 \in A$ sea $\delta > 0$ tal que $B(x_0, \delta) \subset A$. Entonces para cada $u$ y $v \in B(0, \frac{\delta}{2}) \setminus \{0\}$ existen $\alpha, \beta \in (0,1)$ talesque
    $$f\bigl(x_{0} + u + v\bigr) \;-\; f\bigl(x_{0} + u\bigr) \;-\; \bigl(f\bigl(x_{0} + v\bigr) - f(x_{0})\bigr)
    \;=\; D_{v}\bigl(D_{u}f\bigr)\bigl(x_{0} + \alpha\,u + \beta\,v\bigr). $$
\end{lema}
\begin{proof}
    Consideremos la función
    $$g: B(0, \frac{\delta}{2}) \to \mathbb{R}$$
    $$g(x) = f(x + v) - f(x)$$
    Esta funcion $g$ es diferenciable en $B(x_0, \frac{\delta}{2})$ y $dg(x) = df(x + v) - df(x)$. Apliquemos el teorema del valor medio a $g$ en el segmento $L[x_0, x_0 + u]$ y obtenemos la existencia de un punto $c$ entre $x_0$ y $x_0 + u$ tal que
    $$g(x_0 + u) - g(x_0) = dg(c)(u)$$
    Esto es, 
    $$f(x_0 + u + v) - f(x_0 + u) - (f(x_0 + v) - f(x_0)) = (df(c + v) - df(c))(u)$$
    Si escribirmos $c = x_0 + \alpha u$ para algún $\alpha \in (0,1)$, tenemos que
    $$f(x_0 + u + v) - f(x_0 + u) - (f(x_0 + v) - f(x_0)) = (df(x_0 + \alpha u + v) - df(x_0 + \alpha u))(u) = D_uf(x_0 + \alpha u + v) - D_uf(x_0 + \alpha u)$$
    Ahora como $D_uf$ es diferenciable en $A$ y $L[x_0 + \alpha u, x_0 + \alpha u + v] \subset A$, aplicamos el teorema del valor medio a $D_uf$ y nos da que existe un punto $e \in L[x_0 + \alpha u, x_0 + \alpha u + v]$ tal que
    $$D_uf(x_0 + \alpha u + v) - D_uf(x_0 + \alpha u) = d(D_uf(e))(v) = D_{v}(D_uf)(e)$$
    Si escribimos $e = x_0 + \alpha u + \beta v$ para algún $\beta \in (0,1)$, tenemos el resultado
\end{proof}

\begin{teorema}[Teorema de Schwarz]
    Sea $A \subset \mathbb{R}^n$ un abierto y $f \in C^2(A)$, entonces $D_{ij}f(x) = D_{ji}f(x)$ para todo $x \in A$ y para todo $i, j = 1, \ldots, n$. 
\end{teorema}
\begin{proof}
    Sea $x_0 \in A$. Probaremos que $\forall \epsilon > 0 |D_{ij}f(x_0) - D_{ji}f(x_0)| < \epsilon$. Como $D_{ij}f$ y $D_{ji}f$ son continuas en $x_0$, existe $\delta > 0$ tal que $B(x_0, \delta) \subset A$ y 
    $$\|x -x_0\| < \delta \implies \begin{cases}
        |D_{ij}f(x) - D_{ij}f(x_0)| < \epsilon/2 \\
        |D_{ji}f(x) - D_{ji}f(x_0)| < \epsilon/2
    \end{cases}$$
    Tomemos $u = te_i$ y $v = se_j$ con $t, s \in (0, \frac{\delta}{2})$. Tenemos así puntos que están en las condiciones del lema anterior, aplicando ese lema obtenemos que existen $\alpha_1, \beta_1, \alpha_2, \beta_2 \in (0,1)$ tales que
    $$\begin{cases}
            f(x_0 + te_i + se_j) - f(x_0 + te_i) - (f(x_0 + se_j) - f(x_0)) = D_{se_j}(D_{te_i}f)(x_0 + \alpha_1 te_i + \beta_1 se_j)\\
            f(x_0 + se_j + te_i) - f(x_0 + se_j) - (f(x_0 + te_i) - f(x_0)) = D_{te_i}(D_{se_j}f)(x_0 + \alpha_2 se_j + \beta_2 te_i)
    \end{cases}$$
    Como en general $D_{\lambda u}f(x) = \lambda D_u f(x)$ y $A = B$, tenemos que 
    $$st D_{ij}f(x_0 + \alpha_1 te_i + \beta_1 se_j) = ts D_{ji}f(x_0 + \alpha_2 se_j + \beta_2 te_i)$$
    Entoncesporla desigualdad triangular tenemos que
    $$|D_{ij}f(x_0) - D_{ji}f(x_0)| \leq |D_{ij}f(x_0) - D_{ij}f(x_0 + \alpha_2 se_j + \beta_2 te_i)| + |D_{ji}f(x_0 + \alpha_2 se_j + \beta_2 te_i) - D_{ji}f(x_0)|$$
    y tal como habíamos tomado $s$ y $t$ esta suma esmenor que $\epsilon$ pues $\|\alpha_2 s e_j +\beta_2 t e_i\| < s + t < \delta$ y $\|\alpha_1 t e_i + \beta_1 s e_j\| < s + t < \delta$. Por tanto, la arbitrariedad de $\epsilon$ nos da que $D_{ij}f(x_0) = D_{ji}f(x_0)$. 
\end{proof}
\begin{observación}
    \begin{enumerate}
        \item Si $f$ es de clase $C^1$ y existe $\frac{\partial^2 f}{\partial x \partial y}$ y es continua entonces también existe $\frac{\partial^2 f}{\partial y \partial x}$ y coinciden.
        \item También se puede obtener la igualdad entre $D_{12}$ y $D_{21}$ a partir de la existencia de $D_1f$ y $D_2f$ en un entorno de $(x_0, y_0)$ y su diferenciabilidad.
    \end{enumerate}
\end{observación}

\begin{corolario}
    Si $f \in C^3(A)$ entonces $D_{ijk}f(x) = D_{\sigma(i), \sigma(j), \sigma(k)}f(x)$ para todo $x \in A$ y toda permutación $\sigma$ de $\{i, j, k\}$.
\end{corolario}
\begin{proof}
    Supongamos que $\sigma$ está dada por $\sigma(i) = k, \sigma(j) = i, \sigma(k) = j$. 
    \begin{align*}
    D_{ijk}f(x) 
        &= D_i(D_j(D_k f))(x) \\
        &= D_i(D_{jk} f)(x) = D_i(D_{kj} f)(x) \\
        &= D_i(D_k(D_j f))(x) = D_{ik}(D_j f)(x) = D_{ki}(D_j f)(x) = D_{kij} f(x).
\end{align*}
    Las $Dpf$ son de clase $C^2$ y se les puede aplicar el teorema de Schwarz.
\end{proof}

\begin{teorema}[Teorema de Taylor]
    Sea $f: A \subset \mathbb{R}^n \to \mathbb{R}$ una función de clase $C^{k+1}$ en $A$ y sean $x_0, \in A$y $h \neq 0$ tales que $L[x_0, x_0 + h] \subset A$. Entonces existe un punto $c \in L[x_0, x_0 + h]$ tal que
    \begin{align*}
    f(x_0 + h) &= f(x_0) + \sum_{i_1=1}^n D_{i_1}f(x_0) h_{i_1} \\
    &\quad + \frac{1}{2!} \sum_{i_1=1}^n \sum_{i_2=1}^n D_{i_1 i_2}f(x_0) h_{i_1} h_{i_2} + \cdots \\
    &\quad + \frac{1}{k!} \sum_{i_1=1}^n \cdots \sum_{i_k=1}^n D_{i_1 \cdots i_k}f(x_0) h_{i_1} \cdots h_{i_k} \\
    &\quad + \frac{1}{(k+1)!} \sum_{i_1=1}^n \cdots \sum_{i_{k+1}=1}^n D_{i_1 \cdots i_k i_{k+1}}f(c) h_{i_1} \cdots h_{i_k} h_{i_{k+1}},
    \end{align*}
    o equivalentemente usando $h = (x - x_0)$
    \begin{align*}
    f(x) &= f(x_0) + \sum_{i_1=1}^n D_{i_1}f(x_0) (x_{i_1} - x_{0 i_1}) \\
    &\quad + \frac{1}{2!} \sum_{i_1=1}^n \sum_{i_2=1}^n D_{i_1 i_2}f(x_0) (x_{i_1} - x_{0 i_1})(x_{i_2} - x_{0 i_2}) + \cdots \\
    &\quad + \frac{1}{k!} \sum_{i_1=1}^n \cdots \sum_{i_k=1}^n D_{i_1 \cdots i_k}f(x_0) (x_{i_1} - x_{0 i_1}) \cdots (x_{i_k} - x_{0 i_k}) \\
    &\quad + \frac{1}{(k+1)!} \sum_{i_1=1}^n \cdots \sum_{i_{k+1}=1}^n D_{i_1 \cdots i_k i_{k+1}}f(c) (x_{i_1} - x_{0 i_1}) \cdots (x_{i_k} - x_{0 i_k})(x_{i_{k+1}} - x_{0 i_{k+1}}).
    \end{align*}    
\end{teorema}
\begin{proof}
    Sea la aplicación $\varphi:\mathbb{R} \to \mathbb{R}^n$ definida por $\varphi(t) = x_0 + th$. Dado que $\varphi$ es continua entonces $B = \varphi^{-1}(A) \subset \mathbb{R}$ es abierto. \\\\
    Consideremos la composicion $g =  f \circ \varphi: B \subset \mathbb{R} \to \mathbb{R}$, definida por $g(t) = f(x_0 + th)$, que es de clase $C^{k+1}$ en $B$ (pues es una función polinómica y por tanto es $f$ la que restringe la clase). Además se tienq eue $varphi'(t) = h$. Entonces aplicando la regla de la cadena a $\varphi$ obtenemos que: 
    $$g'(t) = df(\varphi(t)) \circ d\varphi(t) = \sum_{i_1 = 1}^{n} \frac{\partial f}{\partial x_{i_1}}(\varphi(t)) \cdot \frac{\partial \varphi}{\partial t}(t) = \sum_{i_1 = 1}^{n} D_{i_1}f(\varphi(t)) h_{i_1} = \langle \nabla f(\varphi(t)), h \rangle$$
    $$ g^{j)} = \sum_{i_1 = 1, \ldots, i_j = 1}^{n} D_{i_1 \ldots i_j}f(\varphi(t)) \cdot h_{i_1} \cdots h_{i_j}$$
    Si aplicamos nuevamente la regla de la cadena a $\varphi$ obtenemos que: 
    $$\frac{\partial}{\partial t} \nabla f(\varphi(t)) = H_f(\varphi(t)) \cdot \frac{\partial \varphi}{\partial t}(t) = H_f(\varphi(t)) \cdot h$$
    Entonces tendríamos que $$g''(t) = \langle \langle H_f(\varphi(t)), h \rangle, h \rangle = H_f(\varphi(t)) \cdot h \cdot h = \sum_{i_1 = 1, i_2 = 1}^{n} \frac{\partial^2 f}{\partial x_{i_1} \partial x_{i_2}}(\varphi(t)) h_{i_1} h_{i_2}$$
    Entonces por inducción podemos llegar a la derivada $\varphi^{j)}$:
    $$g^{j)}(t) = \sum_{i_1 = 1, \ldots, i_n = 1}^{n} D_{i_1 \ldots i_j}f(\varphi(t)) h_{i_1} \cdots h_{i_j}$$
    Podemos aplicar el Teorema de Taylor para una variable una variable real $\varphi$ centrado en $a$: 
    $$f(x) = \sum_{i = 0}^{k} \frac{f^{i)}(a)(x - a)}{i!}$$
    Tomando $a = 0$ como centro y $x = 1$ y además tomaremos la f´romula del resto del valor medio del resto $R_k(x) = \frac{f^{k+1}(\alpha)}{(k+1)!}(x - a)^{k+1}$, donde $\alpha \in (0,1)$, entonces tenemos que: 
    $$g(1) = g(0) + \frac{1}{2!} g''(0) + \cdots + \frac{1}{k!} g^{(k)}(0) + R_k(1) = $$
    $$ = f(x_0) + \sum_{i_1=1}^n D_{i_1}f(x_0) h_{i_1} + \frac{1}{2!} \sum_{i_1=1}^n \sum_{i_2=1}^n D_{i_1 i_2}f(x_0) h_{i_1} h_{i_2} + \cdots + \frac{1}{k!} \sum_{i_1=1}^n \cdots \sum_{i_k=1}^n D_{i_1 \cdots i_k}f(x_0) h_{i_1} \cdots h_{i_k} + R_k(1)$$

\end{proof}
\begin{definición}[Extremos relativos y absolutos]
    Diremos $f: A \subset \mathbb{R}^n$ y $x_0 \in A$. Diremos que $f$ tiene un máximo relativo en $x_0$ si existe $r > 0$ tal que $B(x_0, r) \subset A$ y $f(x) \leq f(x_0)$ para todo $x \in B(x_0, r)$. Si existe una bola centrada en $x_0$ en la $f(x) \leq f(x_0)$ diremos que $f$ tiene un mínimo relativo en $x_0$. Cuando se da alguna de esas desigualdades para todo $x \in A$ diremos que $f$ tiene un máximo absoluto en $x_0$ y un mínimo absoluto en $x_0$.
\end{definición}

\begin{definición}[Punto crítico]
    Si $f$ tiene todas las derivadas parciales de primer orden en un punto $x_0$ diremos que $x_0$ es un \textbf{punto crítico} de $f$ si $\nabla f(x_0) = 0$.
\end{definición}


\begin{teorema}
    Si $f: A \subset \mathbb{R}^n \to \mathbb{R}$ tiene un extremo relativo en $x_0 \in A$ y $f$ tiene todas las derivadas parciales en ese punto entonces $\nabla f(x_0) = 0$.
\end{teorema}
\begin{proof}
    $f$ tiene un máximo relativo en $x_0$, entonces
    $$\frac{f(x_0 +t e_i) - f(x_0)}{t} \leq 0 \quad \forall t > 0$$
    Dado que existe el límitede ese cociente, necesariamente tiene que ser 0. 
\end{proof}
\begin{definición}[Punto de ensilladura]
    Los puntos críticos de $f$ que no son ni máximos ni mínimos relativos se denominan \textbf{puntos de ensilladura} o \textbf{puntos de silla}. 
\end{definición}
\begin{definición}[Matriz hesiana]
    Sea $f: A \subset \mathbb{R}^n \to \mathbb{R}$ una función de clase $C^2$ en $A$. Entonces la \textbf{matriz hesiana} de $f$ en un punto $x_0 \in A$ es la matriz cuadrada de orden $n$ cuyas entradas son las derivadas parciales segundas de $f$:
    $$H_f(x_0) = \begin{pmatrix}
        D_{11}f(x_0) & D_{12}f(x_0) & \cdots & D_{1n}f(x_0)\\
        D_{21}f(x_0) & D_{22}f(x_0) & \cdots & D_{2n}f(x_0)\\
        \vdots & \vdots & \ddots & \vdots\\
        D_{n1}f(x_0) & D_{n2}f(x_0) & \cdots & D_{nn}f(x_0)
    \end{pmatrix}$$
\end{definición}

\begin{definición}[Definición de signo]
    Se dice que una forma cuadrática $Q: \mathbb{R}^n \to \mathbb{R}$ es \textbf{definida positiva} si $Q(x) > 0$ para todo $x \neq 0$. Se dice que es \textbf{definida negativa} si $Q(x) < 0$ para todo $x \neq 0$. Se dice que es \textbf{indefinida} si no es ni positiva ni negativa.
    Se dice que una forma cuadrática es \textbf{semidefinida positiva} si $Q(x) \geq 0$ para todo $x \in \mathbb{R}^n$. Se dice que es \textbf{semidefinida negativa} si $Q(x) \leq 0$ para todo $x \in \mathbb{R}^n$.
\end{definición}
\begin{teorema}
    Sea $f$ una función de clase $C^2$ en un abierto de $\mathbb{R}^n$ y supongamos que un punto $x_0$ de él es un punto crítico para $f$.
    \begin{itemize}
        \item[(a)] Si la forma cuadrática $Q$ asociada a $H_f(x_0)$ es definida negativa, entonces $f$ tiene en $x_0$ un máximo relativo.
        \item[(b)] Si $f$ tiene un máximo relativo en $x_0$, entonces $Q$ es semidefinida negativa.
        \item[(c)] Si $Q$ es definida positiva, entonces $f$ tiene en $x_0$ un mínimo relativo.
        \item[(d)] Si $f$ tiene un mínimo relativo en $x_0$, entonces $Q$ es semidefinida positiva.
        \item[(e)] Si $Q$ es indefinida entonces la función tiene un punto de ensilladura en $x_0$.
    \end{itemize}
\end{teorema}
\begin{proof}
    \begin{enumerate}
        \item[(a)] Como el conjunto $K = \{x \in \mathbb{R}^n : \|x\| = 1\}$ es un conjunto compacto y por tanto acotado, existe $\tilde{x} \in K$ tal que
        $$Q(\tilde{x}) \geq Q(x) \quad \forall x \in K$$
        Ya que al ser $Q$ una aplicación continua y $K$ un compacto, en él debe alcanzar un máximo. Por otro lado al ser $Q$ definida negativa, tenemos que $Q(\tilde{x}) < 0$, por tanto, sea $\epsilon_0 = -Q(\tilde{x})$. 
        Tenemos que por la continuidad de las segundas derivadas de $f$ en $x_0$, existe $\delta > 0$ tal que $B(x_0, \delta) \subset A$ y 
        $$\|x - x_0\| < \delta \implies |D_{ij}f(x) - D_{ij}f(x_0)| < \frac{\epsilon_0}{2n^2} \quad \forall i, j = 1, \ldots, n$$
        Por otra parte, para $h \in \mathbb{R}^n : h \neq 0$ se verifica que: 
        $$Q(h) = \|h\|^2 Q(\frac{h}{\|h\|}) \leq \|h\|^2 Q(\tilde{x}) = -\epsilon_0 \|h\|^2$$
        Apliquemos ahora el Teorema de Taylor, hasta el grado 1, y obtengamos un $c \in L[x_0, x_0 + h]$ para el resto tal que
        $$f(x_0 + h) - f(x_0) = \frac{1}{2} \sum_{i, j = 1}^{n} D_{ij}f(c) h_i h_j = $$
        Nota: Las primeras derivadas son nulas pues es un punto crítico. \\\\
        $$ = \sum_{i, j = 1}^{n} D_{ij}f(c)h_i h_j + D_{ij}f(x_0)h_i h_j - D_{ij}f(x_0)h_i h_j = \sum (D_{ij}f(c) - D_{ij}f(x_0))h_i h_j + D_{ij}f(x_0)h_i h_j = $$
        $$ = \sum_{i, j = 1}^{n} (D_{ij}f(c) - D_{ij}f(x_0))h_i h_j + Q(h)$$
        Si tomamos la cota de la continuidad anterior y lo aplicamos a $c$, tenemos que: 
        $$ \|c - x_0 \| \leq \|x_0 + h - x_0\| = \|h\| < \delta \implies |D_{ij}f(c) - D_{ij}f(x_0)| < \frac{\epsilon_0}{2n^2}$$
        Con lo que obtenemos que: 
        $$ \sum_{i, j =1}^{n} (D_{ij}f(c) - D_{ij}f(x_0))h_i h_j < \frac{\epsilon_0}{2} \|h\|^2$$
        Además, como por otra parte tenemos que 
        $$Q(h) \leq -\epsilon_0 \|h\|^2$$ 
        Obtenemos finalmente que
        $$f(x_0 + h) - f(x_0) = \frac{1}{2} \sum_{i, j =1}^{n} D_{ij}f(x)h_i h_j \leq \frac{1}{2} (-\epsilon_0 + \frac{\epsilon_0}{2}) \|h\|^2 < 0$$
        Entonces obtenemos que $f(x) < f(x_0) \forall x \in B(x_0, \delta)$ y finalmente que $f$ tiene un máximo local en $x_0$. 
        \item[(b)] Demostremos esto por medio del absurdo: \\
        Si existe $h_0$ tal que $Q(h_0) > 0$ consideremos la función $g(t) = -f(x_0 + th_0) = -f(\varphi(t))$ siendo $\varphi(t) = x_0 + th_0$. Se verific que $q$ es de clase $C^2$ en un intervalo centrado en $0$
        $$g'(t) = - (f \circ \varphi)'(t) = -\sum_{i = 1}^{n} D{i}f(x_0 + th_0)h_{0_i} = - \sum_{i = 1}^{n} (D_if \circ \varphi)(t)h_{0_i} = \langle \nabla f(x_0 + th_0), h_0 \rangle$$
        $$g''(t) = - (f \circ \varphi)''(t) = - \sum_{i = 1}^{n} (D_if \circ \varphi)'(t)h_{0_i} = - \sum_{i = 1}^{n} \left(\sum_{j = 1}^{n} D_{j}(D_{i}f)(\varphi(t))h_{0_j}\right)h_{0_i}$$
        Entones, cómo $g'(0) = 0$ (por ser un punto crítico) y $g''(0) = -Q(h_0) < 0$, tenemos que $g$ es cóncava hacia abajo, entonecs $g$ decrece cerca de $t = 0 \implies$
        $$-f(x_0 + th_0) < -f(x_0) \quad \forall t \in (-\delta, \delta)$$
        por lo que $f$ no tendría un máximo local en $x_0$. 
    \end{enumerate}
\end{proof}

\begin{teorema}
    Sea $f: A \subset \mathbb{R}^2 \to \mathbb{R}$ un función de clase $C^2$ y $(x_0, y_0) \in A$ un punto crítico e $f$ y sea $\Delta = det(H_{f}(x_0, y_0)) = ac - b^2$, tenemos que: 
    \begin{enumerate}
        \item Si $\Delta > 0$ y $a > 0$ entonces $f$ tiene un mínimo relativo en $(x_0, y_0)$
        \item Si $\Delta > 0$ y $ a < 0$ entonces $f$ tiene un máximo relativo en $(x_0, y_0)$
        \item Si $\Delta < 0$ entonces $f$ tiene un punto de silla en $(x_0, y_0)$
        \item Si $\Delta = 0$ entonces no se sabe nada
    \end{enumerate}
\end{teorema}
\begin{proof}
    Para la forma cuadrática $Q$ asociada a $H_f(x_0, y_0)$ se verifica que: 
    $$Q(h_1, h_2) = ah_1^2 + 2bh_1h_2 + ch_2^2 $$
    Si $\Delta > 0$ entonces $a \neq 0$ pues $\Delta = ac - b^2$ por lo que: 
    $$Q(h_1, h_2) = ah_1^2 + 2bh_1h_2 + ch_2^2 = a \left(h_1 + \frac{b}{a}h_2\right)^2 + \frac{\Delta}{a}h_2^2$$
    \begin{enumerate}
        \item Si $a > 0 \implies Q(h_1, h_2) > 0 \quad \forall h \in \mathbb{R}^2 \implies f$ tiene un mínimo relativo en $(x_0, y_0)$.
        \item Si $a < 0 \implies Q(h_1, h_2) < 0 \quad \forall h \in \mathbb{R}^2 \implies f$ tiene un máximo relativo en $(x_0, y_0)$.
        \item Si $\Delta < 0$, haremos una distinción de casos: 
        \begin{enumerate}
            \item $a = 0 \implies Q(h_1, h_2) = 2bh_1h_2 + ch_2^2$, y tendríamos que $b \neq 0$ pues si $b = 0$ y $a = 0$ tendríamos que $\Delta = 0$ y hacemos nuevamente distinción de casos: 
            \begin{enumerate}
                \item $c = 0 \implies Q(h_1, h_2) = 2bh_1h_2 \implies \begin{cases}
                    Q(1, 1) = 2b \\
                    Q(1, -1) = -2b
                \end{cases}$
                \item $c \neq 0 \implies Q(h_1, h_2) = 2bh_1h_2 + ch_2^2 \implies \begin{cases}
                    Q(\frac{-c}{b}, 1) = -c \\
                    Q(\frac{-c}{4b}, 1) = \frac{c}{2}            
                \end{cases}$
            \end{enumerate}
            Estos dos casos nos dan que $Q$ toma valores opuestos en diferentes puntos, es decir, $Q$ es indefinida
        \end{enumerate}
        \item $a \neq 0 \implies \begin{cases}
            Q(1, 0) = a \\
            Q(\frac{-b}{a}, 1) = \frac{\Delta}{a} < 0
        \end{cases} \implies$ $Q$ toma valores opuestos en diferentes puntos, es decir, $Q$ es indefinida.
        \item Ni podemos decir nada pues depende de cada caso particular
    \end{enumerate}
\end{proof}

\begin{teorema}[Criterio de Sylvester]
    \begin{enumerate}
        \item Si $\Delta_k > 0 \quad \forall k = 1, \ldots, n$ y $f$ tiene un mínimo relativo en $x_0$ 
        \item Si $(-1)^k \Delta_k > 0 \quad \forall k = 1, \ldots, n$ y $f$ tiene un máximo relativo en $x_0$
        \item Si $\Delta \neq 0$ entonces no hay extremo
    \end{enumerate}
\end{teorema}

\begin{observación}
    Las funcionesdemásdeunavariablepuedentenerun único punto crítico y éstepuedeser un extremo relativo y no absoluto. Por ejemplo, la función$$f(,y) =y^2 +x^2(1+ y)^3$$
    tiene un único punto crítico en $(0, 0)$ y es un mínimo relativo pero no absoluto, ya que $f(0,0) = 0$ y $f(1, -4) = -9$
\end{observación}

\begin{proposición}
    Sea $f: U \subset \mathbb{R}^n \to \mathbb{R}^n$ k-lipschitz con $k \in (0,1) \iff \forall x, y \in U \implies \|f(x) - f(y)\| \leq k \|x - y\|$. \\
    Entonces, la función $g = id +f$ es un homeomorfismo de $U$ en $V$-abierto de $\mathbb{R}^n$
    $$g: U \to V$$
    $$g^{-1}: V \to U$$
\end{proposición}

\begin{teorema}
    Sea $f: \overline{B(x_0, r)} \subset \mathbb{R}^n \to \mathbb{R}^n$ es una aplicación continua en $\overline{B(x_0, r)}$ y y difereniable en $B(x_0, r)$. Si $det(J_f(x)) \neq 0$ para todo $x \in B(x_0, r)$ y $f(x) \neq f(x_0) \forall x : \|x - x_0\| = r$, entonces $f(x_0)$ es un punto interior de $f(B(x_0, r))$
\end{teorema}
\begin{proof}
    Consideremos la función
    $$\varphi: x \in \overline{B(x_0, r)} \subset \mathbb{R}^n \mapsto \varphi(x) = \|f(x) - f(x_0)\|$$
    Como $\varphi$ es continua (ya que $f$ es continua en la bola) y como la frontera de la bola es compacta, entonces $\exists x^* \in \{x : \|x -x_0\| = r\}$ tal que: 
    $$\varphi(x^*) = \min_{x \in \partial B(x_0, r)} \{ \|f(x)-f(x_0)\|\}$$
    Entonces, como $f(x) \neq f(x_0) \forall x \in \partial B(x_0, r)$, tenemos que $\varphi(x^*) > 0$. 
    Ahora beamos que $B(f(x_0), \frac{m}{2}) \subset f(B(x_0, r))$: \\
    Fijado un $y \in B(f(x_0), \frac{m}{2})$, tomemos la función auxiliar:
    $$\psi: x \in \overline{B(x_0, r)} \subset \mathbb{R}^n \mapsto \psi(x) = \|f(x) - y\|$$
    $\psi$ es continua en el compacto $\overline{B(x_0, r)}$ por lo que existe $x^{**} \in \overline{B(x_0, r)}$ tal que $\psi(x^{**}) = \min_{x \in \overline{B(x_0, r)}} \{ \|f(x) - y\|\}$. Ahora veremos que: 
    \begin{itemize}
        \item 
    \end{itemize}
\end{proof}

\begin{corolario}
    Sea $f:A \subset \mathbb{R}^n \to \mathbb{R}^n$ una aplicación de clase $C^1(A)$ y supongamos que $f$ es inyectiva y que $det(J_f(x)) \neq 0$ para todo $x \in A$. Entonces $f$ es unaaplicación abierta.
\end{corolario}

\begin{proposición}
    Sea $f: A \subset \mathbb{R}^n \to \mathbb{R}^n$ una aplicación de clase $C^1$ en el abierto de $A$ y supongamos que para un punto $x_0 \in A$ se verifica que $det(J_f(x_0)) \neq 0$. Entonces, $\exists r > 0$ tal que $B(x_0, r) \subset A$, $f$ es inyectiva en $B(x_0, r)$ y $det(J_f(x)) \neq 0 \forall x \in B(x_0, r)$
\end{proposición}

\begin{teorema}[Teorema de la función inversa]
    Sea $f: A \to \mathbb{R}^n \in C^1(A, \mathbb{R}^n)$ y supongamos que $x_0 \in A$ se verifica que $det(J_f(x_0)) \neq 0$. Entonces existen entornos abiertos $U$ de $x_0$ y $V$ de $f(x_0)$ y una única aplicación $g: V \subset \mathbb{R}^n \to U \subset \mathbb{R}^n$ tal que:
    \begin{enumerate}
        \item $f(U) = V$
        \item $f$ es inyectiva en $U$
        \item $g \circ f = Id$ en $U$
        \item $g \in C^1(V, \mathbb{R}^n)$
    \end{enumerate}
\end{teorema}

\begin{teorema}[Tereoma de la función implícita]
    Sea $F: A \subset \mathbb{R}^n \times \mathbb{R}^k \to \mathbb{R}^n$ una aplicación de clase $C^1$ en $A$-abierto y supongamos que paratodopunto $(x_0, y_0) \in A$ se verifica que $F(x_0, y_0) = 0$ y que $J_yF(x_0, y_0) \neq 0$. Entonces existen entornos abiertos $Y_0$ de $y_0$ y una única aplicación $G: Y_0 \subset \mathbb{R}^k \to \mathbb{R}^n$ tal que:
    \begin{enumerate}
        \item $G \in C^1(Y_0, \mathbb{R}^n)$
        \item $G(y_0) = x_0$
        \item $F(G(y), y) = 0 \quad \forall y \in Y_0$
        \item Existe un entorno $U \subset A$ de $(x_0, y_0)$ tal que: 
        $$\{(x, y) \in : F(x,y) = 0\}= \{(G(y), y) : y \in Y_0\}$$
    \end{enumerate}
\end{teorema}

\begin{definición}[Puntos regulares]
    Sea $f: A \subset \mathbb{R}^n \to \mathbb{R}^k$ siendo $A$ un abierto, un punto $x_0$ es un punto regular si $f$ es de clase $C^1$ en un entorno abierto de $x_0$  la matriz jacobiana de $f$ en $x_0$ es de rango máximo
\end{definición}

\begin{definición}[Variedad regular]
    Se llama \textbf{variedad regular} de dimensión $m$ en $\mathbb{R}^n$ ($m < n$) a todo subconjunto $M$ de $\mathbb{R}^n$ tal que $\forall x_0 \in M$ existe $A \subset \mathbb{R}^n$ abierto  que contiene a $x_0$ y existe una aplicación $F: A \to \mathbb{R}^{n-m}$, regularen cada punto de $A$, por lo que el rango de $J_F(x_0)$ es $n-m$, tal que:
    $$M \cup A = \{x \in A : F(x) = 0 \in \mathbb{R}^{n-m}\}$$
\end{definición}

\begin{definición}[Extremos relativos condicionados]
    Sean $B$ es un conjunto abierto de $\mathbb{R}^n$ y $f$ una aplicación definidade $B$ en $\mathbb{R}$ y $M$ un subconjunto de $B$, se dice que un punto $x_0 \in M$ es un \textbf{máximo relativo condicionado} de $f$ en $M$ si existe un entorno abierto $U$ de $x_0$ tal que $U \cap M \subset B$ y $f(x) \leq f(x_0)$ para todo $x \in U \cap M$. Se dice que es un \textbf{mínimo relativo condicionado} si se cumple la desigualdad inversa.
\end{definición}


\begin{lema}
    Sea $F:A\subset \mathbb{R}^{n-m} \times \mathbb{R}^m \to \mathbb{R}^{n-m}$ de clase $C^1$ en el abierto de $A$ y sea
    $$ M = \{x = (x_1, \ldots, x_{n-m}, \ldots,x_n) \in : F(x) = 0\}$$
    Entonces para todo $x_0 = (x_{0_1}, \ldots, x_{0_{n-m}}, \ldots, x_{0_n}) \in M$ tal que $det(D_iF_j(x_0))_{i, j = 1, \ldots, n-m} \neq 0$ existe un entorno abierto $Y_0 \subset \mathbb{R}^m$ y existe $\psi: Y_0 \to A$ regular en cada putnod e $Y_0$ tal que $\psi(y_0) = x_0$ y $F(\psi(y)) = 0 \forall y \in Y_0$.
\end{lema}

\begin{teorema}[Teorema de los multiplicadores de Lagrange]
    Sea $f, g_1, \ldots, g_m: A \subset \mathbb{R}^n \to \mathbb{R}$ funciones de clase $C^1$ con $m < n$. Consideremos el conjunto
    $$M = \{x \in A : g_i(x) = 0 \quad i = 1, \ldots, m\}$$
    Supongamos que $x_0 \in M : \nabla g_1(x_0), \ldots, \nabla g_m(x_0)$ son linealmente independientes y $x_0$ es un extremo relativo de $f$ restringido a $M$. Entonces, existen $\lambda_1, \ldots, \lambda_m \in \mathbb{R}$ tales que:
    $$\nabla f(x_0) = \sum_{i=1}^m \lambda_i \nabla g_i(x_0)$$
\end{teorema}
\begin{observación}
    \begin{enumerate}
        \item A los $\lambda_i$ se les llama \textbf{multiplicadores de Lagrange} y a la función $L = f + \sum_{i=1}^m \lambda_i g_i$ se le llama \textbf{función de Lagrange} o \textbf{función langrangiana}.
        El teorema nos dice que debemos buscar $x_0$ tal que: 
        $$\begin{cases}
            \nabla L(x_0) = \nabla f(x_0) + \sum_{i=1}^m \lambda_i \nabla g_i(x_0) = 0 \\
            g_i(x_0) = 0 \quad i = 1, \ldots, m
        \end{cases}$$
        \item El teoremase puede enunciar en mayor generalidad que sobre variedades diferenciales $M$
        \item En todo caso, el conjunto $M$ debe ser muy regular (sin picos)
    \end{enumerate}
\end{observación}